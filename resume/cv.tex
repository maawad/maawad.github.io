%%%%%%%%%%%%%%%%%%%%%%%%%%%%%%%%%%%%%%%%%%%%%%%%%%%%%%%%%%%%%%%%%%%%%%%%
%%%%%%%%%%%%%%%%%%%%%% Simple LaTeX CV Template %%%%%%%%%%%%%%%%%%%%%%%%
%%%%%%%%%%%%%%%%%%%%%%%%%%%%%%%%%%%%%%%%%%%%%%%%%%%%%%%%%%%%%%%%%%%%%%%%

%%%%%%%%%%%%%%%%%%%%%%%%%%%%%%%%%%%%%%%%%%%%%%%%%%%%%%%%%%%%%%%%%%%%%%%%
%% NOTE: If you find that it says                                     %%
%%                                                                    %%
%%                           1 of ??                                  %%
%%                                                                    %%
%% at the bottom of your first page, this means that the AUX file     %%
%% was not available when you ran LaTeX on this source. Simply RERUN  %%
%% LaTeX to get the ``??'' replaced with the number of the last page  %%
%% of the document. The AUX file will be generated on the first run   %%
%% of LaTeX and used on the second run to fill in all of the          %%
%% references.                                                        %%
%%%%%%%%%%%%%%%%%%%%%%%%%%%%%%%%%%%%%%%%%%%%%%%%%%%%%%%%%%%%%%%%%%%%%%%%

%%%%%%%%%%%%%%%%%%%%%%%%%%%% Document Setup %%%%%%%%%%%%%%%%%%%%%%%%%%%%

% Don't like 10pt? Try 11pt or 12pt
\documentclass[10pt]{article}

% The automated optical recognition software used to digitize resume
% information works best with fonts that do not have serifs. This
% command uses a sans serif font throughout. Uncomment both lines (or at
% least the second) to restore a Roman font (i.e., a font with serifs).
%\usepackage{times}
%\renewcommand{\familydefault}{\sfdefault}

% This is a helpful package that puts math inside length specifications
\usepackage{calc}
\usepackage{comment}

% Simpler bibsection for CV sections
% (thanks to natbib for inspiration)
\makeatletter
\newlength{\bibhang}
\setlength{\bibhang}{1em} %1em}
\newlength{\bibsep}
 {\@listi \global\bibsep\itemsep \global\advance\bibsep by\parsep}
\newenvironment{bibsection}%
        {\begin{enumerate}{}{%
%        {\begin{list}{}{%
       \setlength{\leftmargin}{\bibhang}%
       \setlength{\itemindent}{-\leftmargin}%
       \setlength{\itemsep}{\bibsep}%
       \setlength{\parsep}{\z@}%
        \setlength{\partopsep}{0pt}%
        \setlength{\topsep}{0pt}}}
        {\end{enumerate}\vspace{-.6\baselineskip}}
%        {\end{list}\vspace{-.6\baselineskip}}
\makeatother

% Layout: Puts the section titles on left side of page
\reversemarginpar

%
%         PAPER SIZE, PAGE NUMBER, AND DOCUMENT LAYOUT NOTES:
%
% The next \usepackage line changes the layout for CV style section
% headings as marginal notes. It also sets up the paper size as either
% letter or A4. By default, letter was used. If A4 paper is desired,
% comment out the letterpaper lines and uncomment the a4paper lines.
%
% As you can see, the margin widths and section title widths can be
% easily adjusted.
%
% ALSO: Notice that the includefoot option can be commented OUT in order
% to put the PAGE NUMBER *IN* the bottom margin. This will make the
% effective text area larger.
%
% IF YOU WISH TO REMOVE THE ``of LASTPAGE'' next to each page number,
% see the note about the +LP and -LP lines below. Comment out the +LP
% and uncomment the -LP.
%
% IF YOU WISH TO REMOVE PAGE NUMBERS, be sure that the includefoot line
% is uncommented and ALSO uncomment the \pagestyle{empty} a few lines
% below.
%

%% Use these lines for letter-sized paper
\usepackage[paper=letterpaper,
            %includefoot, % Uncomment to put page number above margin
            marginparwidth=1.2in,     % Length of section titles
            marginparsep=.05in,       % Space between titles and text
            margin=1in,               % 1 inch margins
            includemp]{geometry}

%% Use these lines for A4-sized paper
%\usepackage[paper=a4paper,
%            %includefoot, % Uncomment to put page number above margin
%            marginparwidth=30.5mm,    % Length of section titles
%            marginparsep=1.5mm,       % Space between titles and text
%            margin=25mm,              % 25mm margins
%            includemp]{geometry}

%% More layout: Get rid of indenting throughout entire document
\setlength{\parindent}{0in}

\usepackage[shortlabels]{enumitem}

%% Reference the last page in the page number
%
% NOTE: comment the +LP line and uncomment the -LP line to have page
%       numbers without the ``of ##'' last page reference)
%
% NOTE: uncomment the \pagestyle{empty} line to get rid of all page
%       numbers (make sure includefoot is commented out above)
%
\usepackage{fancyhdr,lastpage}
\pagestyle{fancy}
%\pagestyle{empty}      % Uncomment this to get rid of page numbers
\fancyhf{}\renewcommand{\headrulewidth}{0pt}
\fancyfootoffset{\marginparsep+\marginparwidth}
\newlength{\footpageshift}
\setlength{\footpageshift}
          {0.5\textwidth+0.5\marginparsep+0.5\marginparwidth-2in}
\lfoot{\hspace{\footpageshift}%
       \parbox{4in}{\, \hfill %
                    \arabic{page} of \protect\pageref*{LastPage} % +LP
%                    \arabic{page}                               % -LP
                    \hfill \,}}

% Finally, give us PDF bookmarks
\usepackage{color,hyperref}
\definecolor{darkblue}{rgb}{0.0,0.0,0.3}
\hypersetup{colorlinks,breaklinks,
            linkcolor=darkblue,urlcolor=darkblue,
            anchorcolor=darkblue,citecolor=darkblue}

%%%%%%%%%%%%%%%%%%%%%%%% End Document Setup %%%%%%%%%%%%%%%%%%%%%%%%%%%%


%%%%%%%%%%%%%%%%%%%%%%%%%%% Helper Commands %%%%%%%%%%%%%%%%%%%%%%%%%%%%

% The title (name) with a horizontal rule under it
% (optional argument typesets an object right-justified across from name
%  as well)
%
% Usage: \makeheading{name}
%        OR
%        \makeheading[right_object]{name}
%
% Place at top of document. It should be the first thing.
% If ``right_object'' is provided in the square-braced optional
% argument, it will be right justified on the same line as ``name'' at
% the top of the CV. For example:
%
%       \makeheading[\emph{Curriculum vitae}]{Your Name}
%
% will put an emphasized ``Curriculum vitae'' at the top of the document
% as a title. Likewise, a picture could be included:
%
%   \makeheading[\includegraphics[height=1.5in]{my_picutre}]{Your Name}
%
% the picture will be flush right across from the name.
\newcommand{\makeheading}[2][]%
        {\hspace*{-\marginparsep minus \marginparwidth}%
         \begin{minipage}[t]{\textwidth+\marginparwidth+\marginparsep}%
             {\large \bfseries #2 \hfill #1}\\[-0.15\baselineskip]%
                 \rule{\columnwidth}{1pt}%
         \end{minipage}}

% The section headings
%
% Usage: \section{section name}
\renewcommand{\section}[1]{\pagebreak[3]%
    \hyphenpenalty=10000%
    \vspace{1.3\baselineskip}%
    \phantomsection\addcontentsline{toc}{section}{#1}%
    \noindent\llap{\scshape\smash{\parbox[t]{\marginparwidth}{\raggedright #1}}}%
    \vspace{-\baselineskip}\par}

% An itemize-style list with lots of space between items
\newenvironment{outerlist}[1][\enskip\textbullet]%
        {\begin{itemize}[#1,leftmargin=*]}{\end{itemize}%
         \vspace{-.6\baselineskip}}

% An environment IDENTICAL to outerlist that has better pre-list spacing
% when used as the first thing in a \section
\newenvironment{lonelist}[1][\enskip\textbullet]%
        {\begin{list}{#1}{%
        \setlength{\partopsep}{0pt}%
        \setlength{\topsep}{0pt}}}
        {\end{list}\vspace{-.6\baselineskip}}

% An itemize-style list with little space between items
\newenvironment{innerlist}[1][\enskip\textbullet]%
        {\begin{itemize}[#1,leftmargin=*,parsep=0pt,itemsep=0pt,topsep=0pt,partopsep=0pt]}
        {\end{itemize}}

% An environment IDENTICAL to innerlist that has better pre-list spacing
% when used as the first thing in a \section
\newenvironment{loneinnerlist}[1][\enskip\textbullet]%
        {\begin{itemize}[#1,leftmargin=*,parsep=0pt,itemsep=0pt,topsep=0pt,partopsep=0pt]}
        {\end{itemize}\vspace{-.6\baselineskip}}

% To add some paragraph space between lines.
% This also tells LaTeX to preferably break a page on one of these gaps
% if there is a needed pagebreak nearby.
\newcommand{\blankline}{\quad\pagebreak[3]}
\newcommand{\halfblankline}{\quad\vspace{-0.5\baselineskip}\pagebreak[3]}

% Uses hyperref to link DOI
\newcommand\doilink[1]{\href{http://dx.doi.org/#1}{#1}}
\newcommand\doi[1]{doi:\doilink{#1}}

% For \url{SOME_URL}, links SOME_URL to the url SOME_URL
\providecommand*\url[1]{\href{#1}{#1}}
% Same as above, but pretty-prints SOME_URL in teletype fixed-width font
\renewcommand*\url[1]{\href{#1}{\texttt{#1}}}

% For \email{ADDRESS}, links ADDRESS to the url mailto:ADDRESS
\providecommand*\email[1]{\href{mailto:#1}{#1}}
% Same as above, but pretty-prints ADDRESS in teletype fixed-width font
%\renewcommand*\email[1]{\href{mailto:#1}{\texttt{#1}}}

%\providecommand\BibTeX{{\rm B\kern-.05em{\sc i\kern-.025em b}\kern-.08em
%    T\kern-.1667em\lower.7ex\hbox{E}\kern-.125emX}}
%\providecommand\BibTeX{{\rm B\kern-.05em{\sc i\kern-.025em b}\kern-.08em
%    \TeX}}
\providecommand\BibTeX{{B\kern-.05em{\sc i\kern-.025em b}\kern-.08em
    \TeX}}
\providecommand\Matlab{\textsc{Matlab}}

%%%%%%%%%%%%%%%%%%%%%%%% End Helper Commands %%%%%%%%%%%%%%%%%%%%%%%%%%%

%%%%%%%%%%%%%%%%%%%%%%%%% Begin CV Document %%%%%%%%%%%%%%%%%%%%%%%%%%%%

\begin{document}
\makeheading{\href{https://maawad.github.io/}{Muhammad A. Awad}}

\section{Contact Information}


\newlength{\rcollength}\setlength{\rcollength}{1.4in}%
%
\begin{tabular}[t]{@{}p{\textwidth-\rcollength}p{\rcollength}}
        \email{mawad@ucdavis.edu} \\
\end{tabular}


%\section{Research Interests}

%Mesh generation and optimization, computer graphics, parallel programming,  and computational fluid dynamics.

\section{Education}

\href{https://www.ucdavis.edu}{\textbf{University of California, Davis}},
Davis, CA
\begin{outerlist}

        \item[] Ph.D. Student, \href{https://www.ece.ucdavis.edu/}{ Electrical and Computer Engineering Department}. 2016 to present

        \begin{innerlist}
                \item Advisor:
                \href{https://www.ece.ucdavis.edu/~jowens/}
                {Professor John D. Owens}
        \end{innerlist}

\end{outerlist}
\vspace{.1in}
\href{http://www.alexu.edu.eg/index.php/en/}{\textbf{Alexandria University}},
Alexandria, Egypt
\begin{outerlist}
        \item[] B.S., Naval Architecture and Marine Engineering Department, 2009 - 2013
        %\begin{innerlist}
        %\item Grade: Very good with honors (78.2\%) Ranked 4th.
        %\end{innerlist}

\end{outerlist}

\section{Research Experience}

\textbf{Graduate Student Researcher} \hfill {September 2016 to present}
\begin{innerlist}

        \item[] Electrical and Computer Engineering Department,\\
        University of California, Davis\\
        Supervisor: Professor John D. Owens
\end{innerlist}

\section{Teaching Experience}

\textbf{Part-Time Teaching Assistant} \hfill {July 2014 to August 2016}
\begin{innerlist}

        \item[] Arab Academy for Science, Technology and Maritime Transport,\\
        College of Maritime Transport and Technology\\
        Courses: Ship Design (MM543T) and Naval Architecture (MM241T).
\end{innerlist}


\section{Professional Experience}

\textbf{Research Intern} \hfill {June 2020 to September 2020}
\begin{innerlist}
        \item[] NVIDIA, Santa Clara, CA\\
        Designing and implementing a GPU string data structure.\\
\end{innerlist}

\textbf{Programming Intern} \hfill {July 2017 to September 2017}
\begin{innerlist}
        \item[] Activision Publishing, Redmond, WA\\
        Testing and implementing different shaders for foliage rendering.\\
\end{innerlist}


\section{Publications}
\vspace{-.1275in}
\begin{bibsection}


        \item \textbf{M. A. Awad}, S. Ashkiani, S. D. Porumbescu and J. D. Owens.
        ``Dynamic Graphs on the GPU."
        \emph{Proceedings of the 34th IEEE International Parallel and Distributed Processing Symposium, IPDPS 2020}. May 2020.


        \item \textbf{M. A. Awad}, S. Ashkiani, R. Johnson, M. Farach-Colton and J. D. Owens.
        ``Engineering a High-Performance {GPU} {B}-Tree."
        \emph{Proceedings of the 24th ACM SIGPLAN Symposium on Principles and Practice of Parallel Programming (PPoPP)}. February 2019.


        \item S. A. Mitchell, M. S. Ebeida, \textbf{M. A. Awad}, C. Park, A. Patney, and A. Rushdi. ``Spoke-Darts for High-Dimensional Blue-Noise Sampling." \emph{ACM Transactions on Graphics (TOG)}. July 2018.


        \item \textbf{M. A. Awad}, A. Rushdi, M. A. Abbas, S. A. Mitchell, A. H. Mahmoud, C. L. Bajaj, M. S. Ebeida.  ``All-Hex Meshing of Multiple-Region Domains without Cleanup." \emph{Proceedings 25th International Meshing Roundtable (IMR25)}. September  2016.

        \item M. S. Ebeida, A. Rushdi, \textbf{M. A. Awad}, A. H. Mahmoud, D.-M. Yan, S. English,
        J. D. Owens, C. Bajaj, and S. A. Mitchell.  ``Disk Density Tuning of a Maximal
        Random Packing." \emph{SGP 2016}. June 2016.

        \item M. S. Ebeida, S. A. Mitchell, A. Patney, A. A. Davidson, S. Tzeng, \textbf{M. A. Awad},
        A. H. Mahmoud, and J. D. Owens.  ``“Exercises in High-Dimensional Sampling:
        Maximal Poisson-disk Sampling and k-d Darts."  In Janine Bennett, Fabien Vivodtzev,
        and Valerio Pascucci, editors, \emph{Topological and Statistical Methods for Complex
                Data Tackling Large-Scale, High-Dimensional, and Multivariate Data Sets, Springer.}   June 2014.


        \item M. S. Ebeida, \textbf{M. A. Awad}, X. Ge, A. H. Mahmoud, S. A. Mitchell, P. M.
        Knupp, and L.-Y. Wei. ``Improving Spatial Coverage while Preserving Blue Noise
        of Point Sets." \emph{Computer Aided Design (SIAM GD/SPM 2013).} November 2013.

        \item M. S. Ebeida, A. H. Mahmoud, \textbf{M. A. Awad}, M. A. Mohammed, S. A. Mitchell,
        A. Rand, and J. D. Owens. ``Sifted Disks." \emph{”Computer Graphics Forum (Eurographics
                2013), 32(2)}. May 2013.


\end{bibsection}

%\section{Talk}
%\vspace{-.125in}
%\begin{bibsection}
%    \item \textbf{M. A. Awad}, A. Abdelkader, A. Rushdi,
%J. D. Owens and M. S. Ebeida. ``SphereCloud: Sizing and Sharp %Features by Iterated Local
%Regression." \emph{”14th U.S. National Congress on Computational Mechanics}. July 2017.
%\end{bibsection}


\section{Talks}
\vspace{-.1275in}
\begin{bibsection}
        \item \textbf{M. A. Awad}.
        ``Engineering a High-Performance {GPU} {B}-Tree."
        \emph{NVIDIA}. April 2019.
\end{bibsection}

\section{Service}
\begin{innerlist}
        \item Reviewer for IEEE Transactions on Parallel and Distributed Systems (TPDS), 2019
\end{innerlist}

\section{Technical Skills}



\begin{innerlist}
        \item Programming: C++, CUDA C/C++, QT, OpenGL\@.
        \item Applications: AutoCAD, Paraview, \LaTeX, MATLAB (linear algebra).
        \item Operating Systems: Microsoft Windows, and Linux.
\end{innerlist}

\section{References}

Professor John D. Owens (advisor)

\begin{innerlist}
        \item[]Child Family Professor of Engineering and Entrepreneurship \\
        Electrical and Computer Engineering Department\\
        University of California, Davis \hfill{E-mail: jowens@ece.ucdavis.edu}\\
\end{innerlist}

\halfblankline


Mohamed S. Ebeida
\begin{innerlist}
        \item[] Senior Member of Technical Staff \\
        Center for Computing Research \\
        Sandia National Laboratories \hfill{E-mail: msebeid@sandia.gov}
\end{innerlist}

\halfblankline


\end{document}
